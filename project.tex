\section{Selected Projects}
\subsection{Physical-Ratio-K based Scheduling Protocol (PRKS)}
\cvitem{}{
PRKS guarantees the reliability of a wireless link is no less than requirement in the presence of interference while maximizing throughput.
I single-handedly implemented PRKS, its two variants, and four state-of-the-art protocols from the ground up, on resource-constrained sensors using TinyOS. Through measurement studies in two sensor network testbeds, each consisting of 127+ sensors, we observed that PRKS enables predictably high link reliability (95\% vs 0\% in others) . The source code is publicly available at \url{https://github.com/xhliu/prks}.
}

\subsection{Multi-Timescale Adaptation (MTA) Routing Protocol}
\cvitem{}{
MTA identifies minimal energy paths that can meet probabilistic deadlines of real-time traffic, given the notorious dynamics and uncertainties of path delays in wireless networks.
I managed to independently implement the whole protocol, its seven variants, and four other protocols from scratch on sensors using TinyOS. Two testbeds of 127+ sensors have verified MTA's significant advantages over the state of the art for a variety of settings, improving deadline success ratio by 89\% and reducing transmission cost by a factor of 9.7. Code is at \url{https://github.com/xhliu/mta}.
}

\subsection{Open Source Community Participation}
\cvitem{}{
TinyOS is the de facto operating system for low-power wireless devices, such as those used in sensor networks, personal area networks.
%,ubiquitous computing,  smart buildings, and smart meters
\url{http://www.tinyos.net}. 
%Thanks to its open source nature, I have acquainted myself with many inner workings of OS: communication (two radio stacks, cc2420 and cc2420x), networking (CSMA/CS MAC and network layer), kernel scheduling,  resource arbitration, timer, SPI/UART, watchdog, and sensing. 
Besides active discussions in the mailing list, I have also reported several bugs, including:
\begin{itemize}
	\item Patch accepted: fixing bug to set default tx power in CC2420X, a radio communication stack for chip CC2420. \url{https://github.com/sallai/tinyos-main/commit/974ff870551d6fcc86f44e311dcbfd0fb71dbc94}
	\item Patch accepted: fixing bug in duplicate detection in CTP, the default routing protocol in TinyOS, send queue. \url{https://www.millennium.berkeley.edu/pipermail/tinyos-help/2010-March/045095.html}
\end{itemize}
}
%both the default CC2420 and CC2420X radio stack
%                % arguments 3 to 6 are optional
%%\cventry{year--year}{Job title}{Employer}{City}{}{Description}                % arguments 3 to 6 are optional
%\subsection{Proficiencies}
%\cventry{10 years}{Perl}{}{}{}{
%I usually prefer Perl for non-graphical implementations.
%The word prediction simulations for my dissertation are written in Perl, as well as much of the dissertation support code.
%I also have experience writing web applications with Perl/CGI and DBI.
%}
%\cventry{12 years}{Java}{}{}{}{
%I prefer Java for applications with graphical interfaces, such as user evaluation of word prediction.
%I also tend to use Java for XML processing, such as analysis of log files from dissertation simulations.
%Also, I used Java and the JADE library to implement the vehicle-to-grid coalition server and client.
%}
%\cventry{10 years}{C++}{}{}{}{I use C++ mainly for academic purposes --- teaching second-semester programming/design and third-semester data structures as well as certain graduate classes (graphics).}
%
%\cventry{familiarity}{Javascript, Python, Lua}{}{}{}{I've written small proof-of-concept applications in several other languages. Many of my systems involving web data display use Javascript to accomplish small tasks.
%}
