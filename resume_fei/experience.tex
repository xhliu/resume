\section{Experience}
\cventry{1/12 -- Present}{Statistical Analyst/SAS Programmer}{American Institute for Research/MacroSys LLC}{Washington DC}{}{
National Assessment of Education Progress (NAEP) project: NAEP’s long-term trend data is the largest nationally representative and continuing assessment of America’s student academic progress in various subject areas.
\begin{itemize}
	\item Utilized factor analysis and multiple linear regression with fixed and random effect to predict difficulty of Grade 4 and 8 reading items with Coh-Metrics factors. 
	\item Implemented bootstrap re-sampling method to create 10,000 hypothetical assessment samples to determine Mathematics and History framework content coverage variability. Validated results with the most recent 5 assessments.
	\item Analyzed complex multi-stage survey data, using techniques such as variance estimation using Jackknife, bias analysis using Chi-square test, t-test, multiple comparison, and HLM.
\end{itemize}
}

\cventry{11/10 -- Present}{Statistical Analyst/SAS Programmer}{American Institute for Research/MacroSys LLC}{Washington DC}{}{
Common Core of Data (CCD) survey project: CCD is the national statistical database of all public elementary and secondary schools and school districts, containing data designed to be comparable across all states.
\begin{itemize}
	\item Implemented review tests to ensure data integrity, accuracy and consistency; developed SAS programs with SAS/Base, SAS/Stat, Macro, Proc sql, and ODS for data documentation and reporting.
	\item Conducted multicollinearity diagnostics and identified correlations between CCD items to reduce reporting burden.
	\item Summarized big data (how big) with descriptive statistics and presented them in easily-understood figures and tables.
\end{itemize}
}

\cventry{1/10 -- 9/10}{Research Data Analyst Intern}{Education Data and Policy Center, Academy for Educational Development}{Washington DC}{}{
\begin{itemize}
	\item Extract education indicators from household survey using STATA, SPSS.
	\item Collect and Format data from global website, Checking data reliability.
	\item Query data already housed in the database using MySQL.
	\item Generating graph reports in EXCEL/VBA/Macro.
	\item Participating in research discussion, recommending analytical methodologies, interpreting and presenting analytical results.
	\item Validate data from StatCompiler to EPDC extraction using SAS/ Proc gplot/proc sql, merging data
\end{itemize}
}

%\cventry{1/09 -- 5/09}{Teaching Assistant}{George Washington University}{Washington DC}{}{
%	\begin{itemize}
%		\item Intro. to Statistics, grading.
%		\item Statistical Modeling, Regression Analysis one-on-one tutoring.
%	\end{itemize}
%}

\cventry{11/07 -- 1/08}{Statistical Programmer}{China Agricultural University}{Beijing}{China}{
\begin{itemize}
	\item Collected and analyzed experimental data in SAS.
	\item Analyzed data with PROC FREQ, MEANS, UNIVERIATE, and PLOT.
	\item Extracted data in DATA step from various data sources such as Excel, ACCESS and other PC data formats, SQL databases.
	\item Manipulated large datasets in DATA step by using By-group processing, data combining and merging, and also using PROC SQL.
	\item Produced summary reports by using PROC TABULATE and PROC REPORT, creating and retrieving user-defined formats.
\end{itemize}
}

\cventry{6/07 -- 8/07}{Research Assistant}{China Agricultural University}{Beijing}{China}{
\begin{itemize}
	\item Managing research databases.
	\item Calculating sample size and power for pilot study of grant proposals.
	\item Conducting parametric \& non-parametric data analysis: Generalized Linear Model, MIXED Model, Survival Analysis, Factor Analysis, Structural Equation Model, Repeated Measures.
	\item Imputing missing data and transforming data.
\end{itemize}
}
