\section{Experience}
\cventry{11/10 -- Present}{Research Associate}{American Institute for Research/MacroSys LLC}{Washington DC}{}{
% what is this project about, how does it achieve its goal, what's the result?
National Assessment of Education Progress (NAEP) project
\begin{itemize}
	% too general
	% "Conduct statistical analyses:" what specific analysis/tools/techniques used
	% "longitudinal database:" what concrete data from the database, how large
	% "in SAS:" what SAS features u used
	\item Conduct statistical analyses for technical reports and research studies using NAEP’s large scale longitudinal database in SAS
	% -> performed X, Y, and Z analysis of 200 MB of A data using SAS's Macro; results are incorporated into 5 research studies on blahblah
	\item Conduct technical review of NAEP data file, report and technical documentation for statistical, psychometric methodological accuracy
\end{itemize}
Common Core of Data survey project (CCD)
\begin{itemize}
	\item Create SAS code to review and edit preliminary data collected from the National Center for Education Statistics Common Core of Data nonfiscal Surveys and generated Excel output using SAS/BASE, Macros, PROC SQL and ODS. Imputed missing data item according to the requirement of client
	\item Provide technical support in CCD report development and documentation creation
	\item Conduct Ad-hoc research study and data request using CCD survey data
\end{itemize}
}

\cventry{1/10 -- 9/10}{Research Data Analyst Intern}{Education Data and Policy Center, Academy for Educational Development}{Washington DC}{}{
\begin{itemize}
	\item Extract education indicators from household survey using STATA, SPSS.
	\item Collect and Format data from global website, Checking data reliability.
	\item Query data already housed in the database using MySQL.
	\item Generating graph reports in EXCEL/VBA/Macro.
	\item Participating in research discussion, recommending analytical methodologies, interpreting and presenting analytical results.
	\item Validate data from StatCompiler to EPDC extraction using SAS/ Proc gplot/proc sql, merging data
\end{itemize}
}

\cventry{1/09 -- 5/09}{Teaching Assistant}{George Washington University}{Washington DC}{}{
	\begin{itemize}
		\item Intro. to Statistics, grading.
		\item Statistical Modeling, Regression Analysis one-on-one tutoring.
	\end{itemize}
}

\cventry{11/07 -- 1/08}{Statistical Programmer}{China Agricultural University}{Beijing}{China}{
\begin{itemize}
	\item Collected and analyzed experimental data in SAS.
	\item Analyzed data with PROC FREQ, MEANS, UNIVERIATE, and PLOT.
	\item Extracted data in DATA step from various data sources such as Excel, ACCESS and other PC data formats, SQL databases.
	\item Manipulated large datasets in DATA step by using By-group processing, data combining and merging, and also using PROC SQL.
	\item Produced summary reports by using PROC TABULATE and PROC REPORT, creating and retrieving user-defined formats.
\end{itemize}
}

\cventry{6/07 -- 8/07}{Research Assistant}{China Agricultural University}{Beijing}{China}{
\begin{itemize}
	\item Managing research databases.
	\item Calculating sample size and power for pilot study of grant proposals.
	\item Conducting parametric \& non-parametric data analysis: Generalized Linear Model, MIXED Model, Survival Analysis, Factor Analysis, Structural Equation Model, Repeated Measures.
	\item Imputing missing data and transforming data.
\end{itemize}
}
