\section{Experience}
\cventry{1/12 -- Present}{Statistical Analyst/SAS Programmer}{American Institute for Research/MacroSys LLC}{Washington DC}{}{
National Assessment of Education Progress (NAEP) project: NAEP’s long-term trend data is the largest nationally representative and continuing assessment of America’s student academic progress in various subject areas.
\begin{itemize}
	\item Utilized factor analysis and multiple linear regression with fixed and random effect to predict difficulty of Grade 4 and 8 reading items with Coh-Metrics factors. 
	\item Implemented bootstrap re-sampling method to create 10,000 hypothetical assessment samples to determine Mathematics and History framework content coverage variability; validated results with the most recent 5 assessments.
	\item Analyzed complex multi-stage survey data, using techniques such as variance estimation using Jackknife, bias analysis using Chi-square test, t-test, multiple comparison, and HLM.
\end{itemize}
}

\cventry{11/10 -- Present}{Statistical Analyst/SAS Programmer}{American Institute for Research/MacroSys LLC}{Washington DC}{}{
Common Core of Data (CCD) survey project: CCD is the national statistical database of all public elementary and secondary schools and school districts, containing data designed to be comparable across all states.
\begin{itemize}
	\item Implemented review tests to ensure data integrity, accuracy and consistency; developed SAS programs with SAS/Base, SAS/Stat, Macro, PROC SQL, and ODS for data documentation and reporting.
	\item Summarized up to 27 years of CCD historical data with descriptive statistics; presented their central tendency and dispersion in concise figures and tables.
	\item Conducted multicollinearity diagnostics and identified correlations between CCD items to reduce reporting burden.
\end{itemize}
}

\cventry{1/10 -- 9/10}{Research Data Analyst Intern}{Education Data and Policy Center, Academy for Educational Development}{Washington DC}{}{
\begin{itemize}
	\item Extracted education indicators from household surveys using STATA and SPSS.
	\item Queried large data sets using SQL: e.g., joining tables, aggregating data with GROUP BY, and deleting duplicates.
\end{itemize}
}

%\cventry{1/09 -- 5/09}{Teaching Assistant}{George Washington University}{Washington DC}{}{
%	\begin{itemize}
%		\item Intro. to Statistics, grading.
%		\item Statistical Modeling, Regression Analysis one-on-one tutoring.
%	\end{itemize}
%}

\cventry{11/07 -- 1/08}{Statistical Programmer}{China Agricultural University}{Beijing}{China}{
\begin{itemize}
	\item Conducted univariate analysis, t-test, and ANOVA on data from factorial experimental design and randomized block design. 
	\item Imputed missing data.
\end{itemize}
}

%\cventry{6/07 -- 8/07}{Research Assistant}{China Agricultural University}{Beijing}{China}{
%\begin{itemize}
%	\item Managing research databases.
%	\item Calculating sample size and power for pilot study of grant proposals.
%	\item Conducting parametric \& non-parametric data analysis: Generalized Linear Model, MIXED Model, Survival Analysis, Factor Analysis, Structural Equation Model, Repeated Measures.
%	\item Imputing missing data and transforming data.
%\end{itemize}
%}
