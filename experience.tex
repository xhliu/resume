\section{Experience}
\cventry{4/12 -- 4/14}{Physical-Ratio-K based Scheduling Protocol (PRKS)}{Research Assistant}{Wayne State University}{MI}{
PRKS guarantees the reliability of a wireless link is no less than requirement in the presence of interference while maximizing throughput. I
\begin{itemize}
	\item Implemented, single-handedly , PRKS, its two variants, and four state-of-the-art protocols from the ground up, on resource-constrained sensors using TinyOS. The source code is publicly available at \url{https://github.com/xhliu/prks}.
	\item Carried out measurements in two sensor network testbeds, each consisting of 127+ sensors, and verified that PRKS enables predictably high link reliability (95\% vs 0\% in others).
\end{itemize}
}

\cventry{9/09 -- 3/12}{Multi-Timescale Adaptation (MTA) Routing Protocol}{Research Assistant}{Wayne State University}{MI}{
MTA identifies minimal energy paths that meet probabilistic deadlines of real-time traffic, taming dynamics and uncertainties of path delays in wireless networks. I
\begin{itemize}
	\item Implemented, independently, the whole protocol, its seven variants, and four other protocols from scratch on sensors using TinyOS. Code is at \url{https://github.com/xhliu/mta}.
	\item Verified MTA's significant advantages over the state of the art for a variety of settings, in two testbeds of 127+ sensors, improving deadline success ratio by 89\% and reducing transmission cost by a factor of 9.7.
\end{itemize}	
}

%\cventry{1/10 -- 1/12}{President}{ACM Student Chapter at Wayne State University}{}{}{
%\emph{\textbf{Responsibilities}}: organized programming competitions and invited speakers to give technical talks.
%}

\cventry{2/09 -- 5/09}{LifeCode}{Technical Team Member}{LifeCode Health}{MI}{
LifeCode is a remote health monitoring system. A wearable sensor measures biometric parameters such as heartbeat rate and transmits the data to a mobile phone. The phone processes and displays the data locally, which also uploads it to servers so health care professionals can access it via a web interface. I
\begin{itemize}
	\item Built a Windows mobile phone application in C\# to receive and display real-time heartbeat rates collected by sensors and transmitted to the phone via Bluetooth.
\end{itemize}
}

\cventry{11/08 -- 5/09}{CSC1000 Ticketing System}{System Administrator}{Wayne State University}{MI}{
Students in lab course CSC1000 report machine breakdowns by filing tickets in this web-based system, where IT personnel can track and update the status of each ticket. I
\begin{itemize}
	\item Modified database schema and wrote PHP code to enable users to insert, update, delete, search, and dump tickets via a web interface using the Linux, Apache, MySQL, and PHP (LAMP) stack.
\end{itemize}
}

\cventry{12/07 -- 3/08}{Microsoft Forefront Security (MFS)}{Software Engineer Intern}{Wicresoft Company}{China}{
MFS is a business antivirus software product that can be controlled over the network. I
\begin{itemize}
	\item Wrote test cases and tested MFS on different Windows families, architectures, and languages.
	\item Automated tests using MS-DOS scripting to run MFS on remote machines with various above configurations.
\end{itemize}
}

\cventry{9/06 -- 11/07}{Websites}{Chief Development Officer}{Trinity Studio}{China}{
Trinity Studio is a studio I co-founded with three classmates, building websites for small businesses, government agencies, and universities. I
\begin{itemize}
	\item Designed database schema using SQL Server and Access.
	\item Developed back end using ASP, IIS, and ODBC.
\end{itemize}
}
