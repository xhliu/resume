\section{Experience}
\cventry{4/12 -- 4/14}{Research Assistant}{Wayne State University}{Michigan}{}{
The patent-pending Physical-Ratio-K based Scheduling Protocol (PRKS) is the first-ever practical distributed scheduling algorithm to ensure reliability of wireless communication while maximizing throughput. %, eliminating the fundamental decade-old hidden-terminal issue.
% in wireless communication. is no less than requirement in the presence of interference while maximizing throughput.
\begin{itemize}
	\item Implemented, single-handedly, PRKS, its two variants, and four state-of-the-art protocols from scratch, on sensors with merely \textit{10kB} RAM
	%48kB ROM, 10kB RAM, and 8MHz 16-bit MCU
	using TinyOS. The source code ($\sim$\textbf{\textit{58,000}} lines) is available at \url{https://github.com/xhliu/prks}.
	\item Drove and refined the theoretical closed-loop controller using measurement data
	\item Designed the architecture of PRKS: split code into reusable encapsulated components, grouped components with coupled functionalities, and organized groups in a tree hierarchy.
	\item Debugged, \textit{without gdb}, and deployed these complex distributed protocols in two real-world multi-hop dynamic testbeds of 127+ sensors and fixed non-repeatable and timing-dependent bugs caused by race conditions and inconsistent distributed states.
	\item Increased link reliability from as low as 0\% in the state of the art to 95\% in PRKS.
\end{itemize}
}

\cventry{9/09 -- 3/12}{Research Assistant}{Wayne State University}{Michigan}{}{
Multi-Timescale Adaptation (MTA) Routing Protocol is the best distributed routing algorithm for delivering probabilistic real-time traffic.
% identifies minimal energy paths that meet probabilistic deadlines of real-time traffic, taming dynamics and uncertainties of path delays in wireless networks.
\begin{itemize}
	\item Implemented, independently, MTA, its seven variants, and four other protocols from the ground up in TinyOS. Code ($\sim$\textbf{\textit{16,000}} lines) is at \url{https://github.com/xhliu/mta}.
	\item Modified TinyOS kernels systematically to make code run concurrently and enable real-time computing, including radio communication stack, time synchronization, and resource arbitration.
	\item Increased deadline success ratio by 89\% and reduced transmission cost by a factor of 9.7, shown in two testbeds of 127+ nodes.
\end{itemize}
}

%\cventry{1/10 -- 1/12}{President}{ACM Student Chapter at Wayne State University}{}{}{
%\emph{\textbf{Responsibilities}}: organized programming competitions and invited speakers to give technical talks.
%}
\cventry{12/07 -- 3/08}{Software Engineer Intern}{Wicresoft Company}{Shanghai}{China}{
	Microsoft Forefront Security (MFS) is a business anti-virus software product that can be controlled over the network.
	\begin{itemize}
		\item Automated tests using MS-DOS scripting by partnering with the development team.
		\item Launched unit and end-to-end tests of MFS on remote machines with different Windows families, CPU architectures, and languages; investigated test failures and performed root-cause analysis.
		\item Found bugs and filed detailed and high-quality error reports in the bug tracking database; tracked them and verified they were fixed.
	\end{itemize}
}

\cventry{2/09 -- 5/09}{Technical Team Member}{LifeCode Health}{Michigan}{}{
LifeCode is a remote health monitoring system to compete in Microsoft Imagine Cup.
%A wearable sensor measures biometric parameters such as heartbeat rate and transmits the data to a mobile phone. The phone processes and displays the data locally, which also uploads it to servers so health care professionals can access it via a web interface.
\begin{itemize}
	\item Built a Windows mobile phone application in C\# to receive and display real-time heartbeat rates collected by wearable sensors and transmitted to the phone via Bluetooth.
\end{itemize}
}

\cventry{11/08 -- 5/09}{System Administrator}{Wayne State University}{Michigan}{}{
Students in lab course CSC1000 report machine breakdowns by filing tickets in a web-based system. %, where IT personnel can track and update the status of each ticket.
\begin{itemize}
	\item Designed database schema and implemented the ticketing system using the LAMP stack, enabling users to insert, update, delete, search, and dump tickets online.
\end{itemize}
}

\cventry{9/06 -- 11/07}{Chief Development Officer}{Trinity Studio}{Wuhan}{China}{
Trinity Studio is a studio I co-founded with three classmates, building websites for small businesses, government agencies, and universities.
\begin{itemize}
	\item Designed and implemented database schema using SQL Server and Access.
	\item Developed back end using ASP, IIS, and ODBC.
\end{itemize}
}
